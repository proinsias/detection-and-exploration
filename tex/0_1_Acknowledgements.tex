\begin{acknowledgements}

As those most likely to actually read my entire thesis, I would first like to thank the faculty who served on my candidacy and defense committees.
Thanks to Wal~Sargent and Andrew~Blain for stepping in on short notice, to Tony~Readhead for understanding the lure of extrasolar planets, to Mike~Brown for putting Pluto in its place, and to Re'em Sari for helping to keep Cathy alive under water!
This thesis work could not have succeeded without Lynne~Hillenbrand, who provided all of the benefits of having two advisors without making me try to please two masters.
Finally, I owe a lot to my advisor David~Charbonneau, who first enticed me to the dark side of exoplanets, and kept my spirits up when the prospects of planet discovery looked bleak.
Though he moved 3,000 miles away to be with his wife---the nerve!---he proved that long-distance advising can work.
(Indeed, I still cringe at the sound of our office phone.)

Strange though it is to acknowledge an inanimate object, without the tireless workhorse that is Sleuth, I would not have helped discover three planets, which might have put the brakes on graduating.
Despite a lonely existence on Mount Palomar, Sleuth was very reliable, even when rained on, unless David or I got on a plane!
More importantly, I must thank those at Palomar Observatory, especially Dipali, Jean, Karl, and Rose, who kept me company on my many one-hour visits.
I sincerely thank Robert~Brucato, Michael~Doyle, Karl~Dunscombe, Richard~Ellis,
Brian~Gordon, John~Henning, Linley~Kroll, Steven~Kunsman, Jean~Mueller, Hal~Petrie, Andrew~Pickles, Nick~Scoville, Merle~Sweet, Robert~Thicksten, Greg~van~Idsinga, Richard Wetzel, and Daniel~Zieber for their assistance with the fabrication, operation, and maintenance of the Sleuth instrument.

My discoveries with Sleuth would never have happened without the help of the TrES team and collaborators.
I have never learned so much about astronomy as I did at our May meetings.
My thanks to Roi~Alonso, G\'{a}sp\'{a}r~Bakos, Nairn~Baliber, Tim~Brown, Orlagh~Creevey, Jonathan~Devor, Ted~Dunham, Juan~Belmonte, Hans~Deeg, Gil~Esquerdo, Mark~Everett, Jos\'{e}~Fern\'{a}ndez, Scott~Gaudi, M\'{a}rton~Hidas, Matt~Holman, Luke~Kotredes, G\'{e}za~Kov\'{a}cs, David~Latham, Georgi~Mandushev, Markus~Rabus, Alex Sozzetti, Bob~Stefanik, John~Trauger, Willie~Torres, Russel~White, and Josh~Winn.

Money makes the world go round, and I gratefully acknowledge the financial support of my thesis work with Sleuth from NASA under the grant NNG05GJ29G, issued through the Origins of Solar Systems Program.

I wish to recognize and acknowledge the very
significant cultural role and reverence that the summit of Mauna Kea
has always had within the indigenous Hawaiian community.  I am most
fortunate to have the opportunity to conduct observations from this
mountain.
Part of this work is based on observations made with the Spitzer Space Telescope, which is operated by the Jet Propulsion Laboratory, California Institute of Technology under a contract with NASA.
This research has made use of the SIMBAD database, operated at CDS, Strasbourg, France, and NASA's Astrophysics Data System Bibliographic Services.
This publication also utilizes data products from the Two Micron All Sky Survey, which is a joint project of the University of Massachusetts and the Infrared Processing and Analysis Center/California Institute of Technology, funded by the National Aeronautics and Space Administration and the National Science Foundation.
This research has made use of the USNOFS Image and Catalogue Archive operated by the United States Naval Observatory, Flagstaff Station.

My thanks to Alicia, Chao, Luke, Sean, and Joanna, my classmates.
You helped me survive first year---Amigo's margaritas will never taste as good as they did as I drank them with my fellow sufferers!
And of course Cathy, Elina, Laura, Margaret, Melissa, Milan and Stuartt, who continued the great tradition of getting to know the First Years, though some more than others.
One's fellow graduate students are a tremendous source of advice regarding all aspects of life.
My thanks to all the Caltech Astronomy students that I have known over the years.
A special mention must be made of Micol and Chin for their help choosing the ring, and of Brian and Danielle for their help choosing our wine!
My thanks also to all the Astronomy staff that made my thesis possible, especially to Patrick~Shopbell, Anu~Mahabul, and Cheryl~Southard for keeping my computers working!

During my thesis, I visited David Charbonneau quite often at the Harvard--Smithsonian Center for Astrophysics.
I thank all of the students there for their warm welcome, in particular Cullen, Heather, Jonathan, Jos\'{e}, Lisa, and Manuel.
I also thank Blue and her family for giving me a second home away from home.

Surprisingly, I did find time to meet people outside of Caltech.
Karen, our Good Samaritan, has helped me and Cathy through all of our marathon struggles---I'll take running the LA marathon over finishing a PhD any day!
My quest to complete 26.2 miles of hell was also aided by the presence of the never-say-die Pasadena Pacers.


I treasure my time over the last five years with my extended family, especially the new additions, and my friends in Cork.
Though separated from me by 6,000 miles, they have continued to be an important part of my life.
They have always made time for me during my visits home, and made me feel like I had never left.

I would never have attempted the PhD program at Caltech without the support of my parents, Tom and Vera, in particular their relentless faith that I can do anything that I put my mind to.
They have given so much to me that can never be repaid, and I am proud to call them not only my parents, but two of my best friends.
I know they will always welcome me back in Cork, especially if I fix the computer one more time!

It's great having a older sibling. She is always an inspiration, a listener, a friend.
I thank my big sister, my ``Best Sister'', Bridget, for always being 18 months older than me, and for letting her younger brother boss her around sometimes.
While I struggled to complete my thesis, she managed to finish hers while working a full-time job.
She makes it all look easy.

And of course, my loving thanks to Cathy, who has gotten to know the real me this year.
While preparing for her own defense on the same day as mine, she always kept some energy to bolster me when things got too much for me, and richly deserves being mentioned three times in these acknowledgments.
She knows me so well, the only time I managed to surprise her was the day I asked her to marry me.
Now that our graduate student days are over, I look forward to a new life together, her hand in mine.

\end{acknowledgements}
