% chktex-file 44
\begin{abstract} % abstract, required by Caltech, 350-word limit

I present the discovery of three transiting planets (\tresTwo, \tresThree, and \tresFour) of nearby bright stars made
with the ten-centimeter telescope Sleuth as part of the Trans-atlantic Exoplanet Survey (TrES).  \tresTwo\ is the first
transiting exoplanet detected in the field of view of NASA's {\textit Kepler} mission.  Of the 20 known transiting
exoplanets, \tresThree\ has the second shortest period, facilitating the study of orbital decay and atmospheric
evaporation.  Its visible/infrared brightness makes \tresThree\ an ideal target for observations to determine the
atmospheric composition.  \tresFour\ has the largest radius and lowest density of the known transiting planets.  These
three planets have radii larger than that of Jupiter, and the radius of \tresFour\ significantly exceeds predictions
from models of hot Jupiters, indicating a possible lack of an energy source in these models.

I present the results of \spi\ observations of \tresTwo.  I reject tidal dissipation of eccentricity as an explanation
for the inflated radius, and examine the spectrum for evidence of atmospheric absorption.

I have monitored 19 fields each containing 6,000--36,000 stars for evidence of transits.  I discuss the rejection of six
of my candidate transiting systems from an early field that represent examples of the 67 astrophysical false positives
that I encountered in Sleuth data.  These six false positives highlight the benefit of a multisite survey such as TrES,
and also of comprehensive follow-up of transit candidates.  As a further example, I present the candidate \gscOTE\ from
Sleuth data that was revealed to be a blend of a bright F dwarf and a fainter K-dwarf eclipsing binary.  This candidate
proved nontrivial to reject, requiring multicolor follow-up photometry to produce evidence of the true binary nature of
this candidate.

The yield of planets from transit surveys is not yet well constrained or understood.  There are numerous factors that
affect the predictions such as the amount of correlated photometric noise present in the data.  Here I present an
analysis of my ability to recover fake transits in TrES data. I examined both the automated transit-search algorithm and
my own visual identification process.  I find the recovery rate of my visual analysis to be 87\% for those transit
candidates that had a sufficiently high signal-to-noise ratio to be flagged by my transit-search algorithm and readily
identifiable by eye.

\end{abstract}
