\chapter[Summary]{%
Summary}
\label{cha:discuss}

I have presented the results of my thesis work as part of the Trans-atlantic Exoplanet Survey (TrES) for transiting exoplanets.
Using a ten-centimeter telescope to monitor 19 fields each containing 6,000--36,000 stars, I have identified 67 candidate transiting systems.
To date, three of these have been confirmed by determining the spectroscopic orbit of the host star caused by the planetary companion.
Due to the small number of known transiting exoplanets, this is a noticeable contribution to the field, especially since these transiting systems are bright enough to be within the reach of HST, {\spi}, and future JWST observations.

I rejected the majority of the candidates as astrophysical false positives.
(The status of the remaining candidates is not yet fully determined.)
This rejection was made on the basis of my follow-up photometric observations as well as spectroscopy and photometry obtained by the TrES team.
The experience of the TrES team with efficiently identifying these impostors has resulted in a high yield from the resource-intensive high signal-to-noise ratio, high-precision spectroscopy that I have obtained with Keck/HIRES.

The TrES team have also encountered several blended eclipsing binaries that were quite difficult to identify correctly.
However, as a result of a detailed examination of one-meter class photometry and spectroscopy, we removed these from our list without the need of larger telescopes.
One of these blended systems that I identified from Sleuth data did not show any spectroscopic evidence of the eclipsing binary, but was later shown to have a color-dependent eclipse depth, indicative of its true nature.

I have conducted followup of the planets that I detected.
My analysis of the {\spi} observations of \tresTwo\ concluded that this planet has indeed the expected circular orbit of a hot Jupiter, and that the spectrum shows no clue of the atmospheric composition, although the scheduled additional observations of the planet are critical to confirm this.

Looking toward the future, I will be continuing to observe with Sleuth as part of TrES.
Based on the yield of transiting planets from my thesis work, I expect to detect perhaps as many as three additional planets during the remainder of the survey.
I will obtain additional {\it Spitzer} observations of \tresTwo\ (and hopefully the new TrES planets) during the remaining lifetime of the instrument, and continue to probe the exoplanetary atmospheres.
I will also be completing my analysis of the survey yield, from which I will determine the effect of the human determinant on the recovery of candidate transiting systems.
