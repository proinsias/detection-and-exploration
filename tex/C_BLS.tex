\chapter[The Box-fitting Least-Squares (BLS) Transit-Search Algorithm]%
{%
\LARGE The Box-fitting Least-Squares (BLS) Transit-Search Algorithm%
}
\label{cha:bls}

As a reference to the reader, in this appendix I summarize the method to search for periodic transits signals in photometric data proposed by Kovacs, Zucker, \& Mazeh~(\citeyear{Kovacs_Zucker_Mazeh:aa:2002a}).
For easy comparison with the reference paper, in general I use the authors' notation.

\section{A General Description of the BLS Algorithm}
\label{cha:bls:sec:desc}

When a dark planet transits the disk of a star, the resulting U-shaped light curve (see, e.g., figure~\ref{cha:human:sec:model:fig:agol}) requires a complex analytical description \citep{Mandel_Agol:apjl:2002a}.
Trying to fit such a model to each light curve from a transit survey such as the Trans-atlantic Exoplanet Survey (TrES) would be prohibitively computationally intensive.
Instead, \citet{Kovacs_Zucker_Mazeh:aa:2002a} proposed searching for a periodic transit signal (of period $P_{0}$) using a much simpler box fit, that is assuming a light curve with two discrete levels ($H$ and $L$), with an ingress phase $\phi$, and with a short fractional duration ($q$) at the lower level, $L$.
The value of $q$ is assumed to lie within $[0.01,0.10]$, as expected for almost all transiting planets~\citep*{Defay_Deleuil_Barge:aa:2001a}.
For TrES time series of differential magnitudes, we also assume $H=0$, and hence $L=\delta$, the depth of the transit.
For example of a light curve of a (fake) transit candidate, with the box-fit light curve overplotted, see figure~\ref{cha:human:sec:model:fig:highsde}.

The BLS algorithm takes as input a photometric time series, a range of periods $[P_{1},\ldots, P_{n_{p}}]$ to search, and the number of trial periods ($n_{p}$) and bins ($n_{b}$) to use.
The period step separating each trial period is then $\Delta p=(P_{n_{p}} - P_{1})/n_{p}$.
For a given trial period $P_{i}$, the time series is folded and averaged in the $n_{b}$ bins.
The algorithm then computes the Signal Residue (a measure of the quality of the fit; see below), and the best-fit $q_{i}$, $\delta_{i}$, and $\phi_{i}$ for this trial period.

The best-fit period $\hat{P}$ is the trial period that gives the largest Signal Residue.
The overall best-fit parameters from the BLS algorithm for the input time series are then $\hat{P}$, and the corresponding \{$\hat{q}$, $\hat{\delta}$, $\hat{\phi}$\}.

\section[The Signal Residue and Signal Detection Efficiency]{The Signal Residue and Signal Detection \\ Efficiency}
\label{cha:bls:sec:defs}

Assume we have a time series phased using a trial period $P_{i}$.
Let us denote the folded time series by $\{ \hat{x}_{i} \}, i=1,2,\ldots,n$.
Each data point $\hat{x}_{i}$ has an associated uncertainty $\sigma_{i}$, and a corresponding weight
\begin{eqnarray*}
\hat{w}_{i} & = & \left( \sigma_{i}^{2} \sum_{j=1}^{n} \frac{1}{\sigma_{j}^{2}} \right)^{-1}.
\end{eqnarray*}
Now suppose that in-transit occurs in the bins $(i_{1}, i_{2})$, under the condition that $q = (i_{2} - i_{1})/n_{b}$ lies within the above range. Define
\begin{eqnarray*}
r(i_{1},i_{2}) & = & \sum_{i=i_{1}}^{i_{2}} \hat{w}_{i},
\end{eqnarray*}
that is, the sum of the weights of the in-transit data, and
\begin{eqnarray*}
s(i_{1},i_{2}) & = & \sum_{i=i_{1}}^{i_{2}} \hat{w}_{i} \hat{x}_{i},
\end{eqnarray*}
i.e., the weighted sum of the in-transit data.
The Signal Residue (SR) of the time series for $P_{i}$ is then defined as
\begin{eqnarray*}
\mathrm{SR} & = & \mathrm{MAX} \left\{ \left[ \frac{s^{2}(i_{1},i_{2})}{r(i_{1},i_{2})[1-r(i_{1},i_{2})]} \right]^{\frac{1}{2}} \right\}.
\end{eqnarray*}

The least-squares best-fit for the time series is then computed by finding the period $\hat{P}$ that maximizes SR.
The largest Signal Residue, $\hat{\mathrm{SR}}$, has associated in-transit bins $(\hat{i}_{1}, \hat{i}_{2})$.
The best-fit transit depth $\hat{\delta}$ can then be derived using
\begin{eqnarray*}
\hat{\delta} = \frac{\hat{\mathrm{SR}}}{\sqrt{r(\hat{i}_{1}, \hat{i}_{2})[1-r(\hat{i}_{1}, \hat{i}_{2})]}}.
\end{eqnarray*}
The best-fit ingress phase is $ \hat{i}_{1}/n_{b}$, and the best-fit fractional transit duration is
\begin{eqnarray*}
\hat{q} & = & \frac{\hat{i}_{2} - \hat{i}_{1}}{n_{b}}.
\end{eqnarray*}

Once we have derived the best fit for a given time series, we can compute for that time series the Signal Detection Efficiency (SDE)
\begin{eqnarray*}
\mathrm{SDE} & = &(\hat{\mathrm{SR}}\ - <\mathrm{SR}>)/\sigma(\mathrm{SR}),
\end{eqnarray*}
where $<\mathrm{SR}>$ and $\sigma(\mathrm{SR})$ are the average and standard deviation, respectively, of the Signal Residue over the period range tested.
The SDE can then be used as a comparative measure of the strengths of the transit detections in a sequence of time series, such as the photometry from TrES.
